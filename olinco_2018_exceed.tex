\documentclass[12pt]{beamer}
\begin{filecontents}{\jobname.bib}

@inproceedings{buring2008least,
  title={The least at least can do},
  author={B{\"u}ring, Daniel},
  booktitle={Proceedings of WCCFL},
  volume={26},
  pages={114--120},
  year={2008},
  organization={Citeseer}
}


@phdthesis{solt2009semantics,
  title={The {S}emantics of {A}djectives of {Q}uantity},
  author={Solt, Stephanie},
  year={2009},
  school={City University of New York}
}

@incollection{beck2011comparison,
  title={Comparison constructions},
  author={Beck, Sigrid},
  booktitle={Semantics: {A}n {I}nternational {H}andbook of {N}atural {L}anguage {M}eaning},
  editor={Maienborn, Claudia and von Heusinger, Klaus and Portner, Paul},
  publisher={De Gruyter Mouton},
  pages={1341--1389},
  address={Berlin},
  year={2011}
}

@unpublished{gajewski2002analycity,
  title={L-analycity in natural language},
  author={Gajewski, Jon},
  year={2002},
  note={Ms. at MIT}
}

@article{beavers_koontz-garboden2012manner,
  title={Manner and result in the roots of verbal meaning},
  author={Beavers, John and Koontz-Garboden, Andrew},
  journal={Linguistic {I}nquiry},
  volume={43},
  number={3},
  pages={331--369},
  year={2012},
  publisher={MIT Press}
}

@article{kearns2007telic,
  title={Telic senses of deadjectival verbs},
  author={Kearns, Kate},
  journal={Lingua},
  volume={117},
  number={1},
  pages={26--66},
  year={2007},
  publisher={Elsevier}
}

@unpublished{kennedy2005variation,
  title={Variation in the expression of comparison},
  author={Kennedy, Chris},
  note={Presented at Cornell University, 11.11.2005},
  year={2005}
}

@incollection{kennedy_levin2008measure,
	title={Measure of change: {T}he adjectival core of degree achievements},
	author={Kennedy, Christopher and Levin, Beth},
	booktitle={Adjectives and {A}dverbs: {S}yntax, {S}emantics and {D}iscourse},
	editor={McNally, Louise and Kennedy, Christopher},
	pages={156--182},
	address={Oxford},
	publisher={Oxford University Press},
	year={2008}
}

@incollection{tatevosov2004slavic,
	title={Intermediate prefixes in {R}ussian},
	author={Tatevosov, Sergei},
	booktitle={Annual {W}orkshop on {F}ormal {A}pproaches to {S}lavic {L}inguistics: {T}he {S}tony {B}rook {M}eeting},
	editor={Antonenko, Andrei},
	pages={423--445},
	address={Ann Arbor},
	publisher={Michigan Slavic Publications},
	year={2008}
}

@incollection{janda1985meaning,
	title={The meaning of {R}ussian verbal prefixes: {S}emantics and {G}rammar},
	author={Janda, Laura},
	booktitle={The {S}cope of {S}lavic {A}spect},
	editor={Flier, M.S. and Timberlake, A.},
	pages={26--40},
	address={Columbus},
	publisher={Slavica},
	year={1985}
}

@incollection{rappaport-hovav2008lexicalized,
  title={Lexicalized meaning and the internal structure of events},
  author={Rappaport Hovav, Malka},
  booktitle={Theoretical and {C}rosslinguistic {A}pproaches to the {S}emantics of {A}spect},
  editor={Rothstein, Susan},
  pages={13--42},
  year={2008},
  publisher={John Benjamins Publishing},
  address={Amsterdam},
  year={2008}
}

@incollection{filip2005measure,
  title={Measures and indefinites},
  author={Filip, Hana},
  booktitle={References and {Q}uantification: {T}he {P}artee {E}ffect},
  editor={Carlson, Gregory N. and Pelletier, Francis Jeffry},
  pages={229--288},
  address={Stanford},
  publisher={CSLI Publications},
  year={2005}
}

@incollection{souckova2004there,
  title={There's only one po-},
  booktitle={Nordlyd 32.2: {S}pecial {I}ssue on {S}lavic {P}refixes},
  author={Sou{\v{c}}ko{\'{a}}, Kate{\v{r}}ina},
  editor={Svenonius, Peter},
  pages={403--419},
  address={Troms{\o}},
  publisher={CASTL},
  year={2004}

@incollection{svenonius2004slavic,
  title={Slavic prefixes inside and outside {VP}},
  author={Svenonius, Peter},
  booktitle={Nordlyd 32.2: {S}pecial {I}ssue on {S}lavic {P}refixes},
  editor={Svenonius, Peter},
  pages={205--253},
  address={Troms{\o}},
  publisher={CASTL},
  year={2004}
}

@incollection{ramchand2004time,
  title={Time and the event: {T}he semantics of {R}ussian prefixes},
  author={Ramchand, Gillian},
  booktitle={Nordlyd 32.2: {S}pecial {I}ssue on {S}lavic {P}refixes},
  editor={Svenonius, Peter},
  pages={323--361},
  address={Troms{\o}},
  publisher={CASTL},
  year={2004}
}

@article{nouwen2010two,
  title={Two kinds of modified numerals},
  author={Nouwen, Rick},
  journal={Semantics and {P}ragmatics},
  volume={3},
  pages={1--41},
  year={2010}
}

@inproceedings{nouwen2008directionality,
  title={Directionality in numeral quantifiers: {T}he case of `up to'},
  author={Nouwen, Rick},
  booktitle={Proceedgins from {S}emantics and {L}inguistic {T}heory 18},
  editor={Friedman, Tova and Ito, Satoshi},
  pages={569--582},
  year={2008},
  publisher={CLC Publications},
  address={Ithaca},
}

@incollection{nouwen2015modified,
  title={Modified numerals: {T}he epistemic effect},
  author={Nouwen, Rick},
  booktitle={Epistemic {I}ndefinites},
  editor={Alonso-Ovalle, Luis and Menéndez-Benito, Paula},
  pages={244--266},
  year={2015},
  publisher={Oxford University Press},
  address={Oxford}
}

@incollection{filip2000quantization,
  title={The quantization puzzle},
  author={Filip, Hana},
  booktitle={Events as {G}rammatical {O}bjects, from the {C}ombined {P}erspectives of {L}exical {S}emantics, {L}ogical {S}emantics and {S}yntax},
  editor={Tenny, Carol and Pustejovsky, James},
  pages={39--91},
  year={2000},
  publisher={CSLI Press},
  address={Stanford}
}

@incollection{filip2008events,
  title={Events and maximalization: {T}he case of telicity and perfectivity},
  author={Filip, Hana},
  booktitle={Theoretical and {C}rosslinguistic {A}pproaches to the {S}emantics of {A}spect},
  editor={Rothstein, Susan},
  pages={217--256},
  year={2008},
  publisher={John Benjamins Publishing},
  address={Amsterdam}
}

@book{stassen1985comparison,
  title={Comparison and {U}niversal {G}rammar},
  author={Stassen, Leon},
  year={1985},
  publisher={Basil Blackwell},
  address={Oxford}
}

@phdthesis{rett2008degree,
  title={Degree {M}odification in {N}atural {L}anguage},
  author={Rett, Jessica},
  year={2008},
  school={The State University of New Jersey}
}

@phdthesis{romanova2006constructing,
	title={Constructing {P}erfectivity in {R}ussian},
	author={Romanova, Eugenia},
	year={2006},
	school={University of Troms{\o}}
}

@phdthesis{hackl2001comparative,
	title={Comparative {Q}uantifiers},
	author={Hackl, Martin},
	year={2001},
	school={Massachusetts  Institute  of  Technology}
}

@phdthesis{bochnak2013crosslinguistic,
  title={Cross-linguistic {V}ariation in the {S}emantics of {C}omparatives},
  author={Bochnak, Michael Ryan},
  year={2013},
  school={University of Chicago}
}

@inproceedings{hay_kennedy_levin1999scalar,
  title={Scalar structure underlies telicity in `degree achievements'},
  author={Hay, Jennifer and Kennedy, Christopher and Levin, Beth},
  booktitle={Proceedings from {S}emantics and {L}inguistic {T}heory 9},
  editor={Matthews, Tanya and Strolovitch, Devon},
  pages={127--144},
  address={Ithaca},
  publisher={CLC Publications},
  year={1999}
}

@article{kagan2013scalarity,
  title={Scalarity in the domain of verbal prefixes},
  author={Kagan, Olga},
  journal={Natural {L}anguage \& {L}inguistic {T}heory},
  volume={31},
  number={2},
  pages={483--516},
  year={2013},
  publisher={Springer}
}

@inproceedings{kagan_alexeyenko2011adjectival,
  title={The adjectival suffix -ovat as a degree modifier in {R}ussian},
  author={Kagan, Olga and Alexeyenko, Sascha},
  booktitle={Proceedings of {S}inn und {B}edeutung 15},
  editor={Reich, Ingo},
  pages={321--335},
  year={2011},
  address={Saarbrücken},
  publisher={Universaar}
}

@incollection{kagan2012degree,
  title={Degree semantics for {R}ussian verbal prefixes: {T}he case of pod- and do-},
  author={Kagan, Olga},
  booktitle={The {R}ussian {V}erb: {O}slo {S}tudies in {L}anguage},
  editor={Gr{\o}nn, Atle and Pazel'skaya, Anna},
  volume={4},
  number={1},
  year={2012}
}

@article{kagan2011scale,
  title={The scale hypothesis and the prefixes pere- and nedo-},
  author={Kagan, Olga},
  journal={Scando-{S}lavica},
  volume={57},
  number={2},
  pages={160--176},
  year={2011},
  publisher={Taylor \& Francis}
}

@inproceedings{filip1992aspect,
  title={Aspect and interpretation of nominal arguments},
  author={Filip, Hana},
  booktitle={Papers from the 28th {R}egional {M}eeting of the {C}hicago {L}inguistic {S}ociety},
  editor={Canakis, Costas P. and Chan, Grace P. and Denton, Jeanette Marshall},
  pages={139--158},
  year={1992},
  address={Chicago},
  publisher={University of Chicago}
}

@article{brasoveanu2012modified,
  title={Modified numerals as post-suppositions},
  author={Brasoveanu, Adrian},
  journal={Journal of {S}emantics},
  volume={30},
  number={2},
  pages={155--209},
  year={2012},
  publisher={Oxford University Press}
}

@inproceedings{heim2000degree,
	address={Ithaca, NY},
    title = {Degree operators and scope},
	booktitle = {Proceedings of {S}emantics and {L}inguistic {T}heory 10},
	author = {Heim, Irene},
	year = {2000},
    publisher={CLC Publications},
    editor={Jackson, Brendan and Matthews, Tanya},
	pages = {40-64},
    doi={10.3765/salt.v10i0.3102}
}

@article{kennedy_mcnally2005scale,
  title={Scale structure, degree modification, and the semantics of gradable predicates},
  author={Kennedy, Christopher and McNally, Louise},
  journal={Language},
  volume={81},
  number={2},
  pages={345-381},
  year={2005},
  doi={10.1353/lan.2005.0071}
}

@article{morzycki2009degree,
  title={Degree modification of gradable nouns: {S}ize adjectives and adnominal degree morphemes},
  author={Morzycki, Marcin},
  journal={Natural {L}anguage {S}emantics},
  volume={17},
  number={2},
  pages={175-203},
  year={2009},
  publisher={Springer},
  doi={10.1007/s11050-009-9045-7}
}

@article{kennedy2001polar,
  title={Polar opposition and the ontology of `degrees'},
  author={Kennedy, Chris},
  journal={Linguistics and {P}hilosophy},
  volume={24},
  number={1},
  pages={33-70},
  year={2001},
  publisher={Kluwer Academic Publishers},
  doi={10.1023/A:1005668525906}
}

@article{schwarzschild_wilkinson2002quantifiers,
  title={Quantifiers in comparatives: {A} semantics of degree based on intervals},
  author={Schwarzschild, Roger and Wilkinson, Karina},
  journal={Natural {L}anguage {S}emantics},
  volume={10},
  number={1},
  pages={1-41},
  year={2002},
  publisher={Kluwer Academic Publishers},
  doi={10.1023/A:1015545424775}
}

@inproceedings{schwarzschild2002grammar,
  address={Ithaca, NY},
  title={The grammar of measurement},
  author={Schwarzschild, Roger},
  booktitle={Proceedings of {S}emantics and {L}inguistic {T}heory 12},
  pages={225-245},
  year={2002},
  publisher={CLC Publications},
  editor={Jackson, Brendan},
  doi={10.3765/salt.v12i0.2870}
}

@inproceedings{pancheva2006phrasal,
  title={Phrasal and clausal comparatives in {Slavic}},
  address={Ann Arbor, MI},
  author={Pancheva, Roumyana},
  booktitle={Formal {A}pproaches to {S}lavic {L}inguistics 14: {T}he {P}rinceton {M}eeting 2005},
  editor={Lavine, James and Franks, Steven and Tasseva-Kurktchieva, Mila and Filip, Hana},
  pages={236-257},
  publisher={Michigan Slavic Publications},
  year={2006}
}

@article{solt2015q-adjectives,
  title={Q-adjectives and the semantics of quantity},
  author={Solt, Stephanie},
  journal={Journal of {S}emantics},
  volume={32},
  number={2},
  pages={221-273},
  year={2015},
  doi={10.1093/jos/fft018}
}

@article{rett2015measure,
	title = {Measure phrase equatives and modified numerals},
	journal = {Journal of {S}emantics},
    volume = {32},
    number = {3},
	author = {Rett, Jessica},
	year = {2015},
	pages = {425-475},
    doi={10.1093/jos/ffu004}
}

@article{rett2014polysemy,
	title={The polysemy of measurement},
	volume={143},
	journal={Lingua},
	author={Rett, Jessica},
	year={2014},
	pages={242--266}
}

@article{schwarzschild2008semantics,
	title={The semantics of comparatives and other degree constructions},
	volume={2},
	number={2},
	journal={Language and {L}inguistics {C}ompass},
	author={Schwarzschild, Roger},
	year={2008},
	pages={308--331}
}

@article{von_stechow1984comparing,
	title={Comparing semantic theories of comparison},
	volume={3},
	number={1},
	journal={Journal of {S}emantics},
	author={von Stechow, Arnim},
	year={1984},
	pages={1--77}
}
\end{filecontents}

\setbeamertemplate{caption}[numbered]
\setbeamertemplate{caption label separator}{: }
\setbeamercolor{caption name}{fg=normal text.fg}
\beamertemplatenavigationsymbolsempty
\setbeamertemplate{navigation symbols}{}
\setbeamertemplate{footline}[page number]
\usepackage{lmodern}
\usepackage{amssymb,amsmath}
\usepackage{ifxetex,ifluatex}
\usepackage{fixltx2e} % provides \textsubscript
\ifnum 0\ifxetex 1\fi\ifluatex 1\fi=0 % if pdftex
\usepackage[T1]{fontenc}
\usepackage[utf8]{inputenc}
\else % if luatex or xelatex
\ifxetex
\usepackage{mathspec}
\else
\usepackage{fontspec}
\fi
\defaultfontfeatures{Ligatures=TeX,Scale=MatchLowercase}
\fi

\usepackage{longtable,booktabs}
\usepackage{multirow}
\usepackage{caption}
% These lines are needed to make table captions work with longtable:
\makeatletter
\def\fnum@table{\tablename~\thetable}
\makeatother
\usepackage{graphicx,grffile}
\makeatletter
\def\maxwidth{\ifdim\Gin@nat@width>\linewidth\linewidth\else\Gin@nat@width\fi}
\def\maxheight{\ifdim\Gin@nat@height>\textheight0.8\textheight\else\Gin@nat@height\fi}
\makeatother
% Scale images if necessary, so that they will not overflow the page
% margins by default, and it is still possible to overwrite the defaults
% using explicit options in \includegraphics[width, height, ...]{}
\setkeys{Gin}{width=\maxwidth,height=\maxheight,keepaspectratio}

\usepackage{natbib}
\bibliographystyle{apalike}
% make bibliography entries smaller
\renewcommand\bibfont{\tiny}
% If you have more than one page of references, you want to tell beamer
% to put the continuation section label from the second slide onwards
\setbeamertemplate{frametitle continuation}[from second]
% Now get rid of all the colours
\setbeamercolor*{bibliography entry title}{fg=black}
\setbeamercolor*{bibliography entry author}{fg=black}
\setbeamercolor*{bibliography entry location}{fg=black}
\setbeamercolor*{bibliography entry note}{fg=black}

\setbeamertemplate{frametitle continuation}[from second][]

% and kill the abominable icon
\setbeamertemplate{bibliography item}{}

\setlength{\emergencystretch}{3em}  % prevent overfull lines
\providecommand{\tightlist}{%
	\setlength{\itemsep}{0pt}\setlength{\parskip}{0pt}}
\setcounter{secnumdepth}{0}
\usepackage{linguex}
\renewcommand\eachwordone{\sffamily} %sans serif in glossed examples
\renewcommand\eachwordtwo{\sffamily} %sans serif in glossed examples
\newcommand*{\myfont}{\fontfamily{pcr}\selectfont}
\usepackage{qtree}
\usepackage{stmaryrd}
\usepackage{capt-of}
\usepackage{eurosym}
\usepackage{gensymb}

\title{\textsc{Exceed} comparison in Czech and A/B numeral modifiers}
\author{Mojmír Dočekal, Hana Filip, Marcin Wągiel}
\date{OLINCO 2018, Palacký university Olomouc, June 7--9 2018}

\begin{document}
	\frame{\titlepage}
	
	%[allowframebreaks]

\begin{frame}{Introduction}

Starting point

\begin{itemize}
\item research on comparatives\\
\scriptsize von Stechow (\citeyear{von_stechow1984comparing}), Heim (\citeyear{heim2000degree}), Kennedy \& McNally (\citeyear{kennedy_mcnally2005scale}), Pancheva (\citeyear{pancheva2006phrasal}), Schwarzschild (\citeyear{schwarzschild2008semantics}), Rett (\citeyear{rett2008degree}), Solt (\citeyear{solt2009semantics})\normalsize
\vspace{0.666em}
\begin{itemize}
\item understudied: \textit{exceed} comparison\\
\scriptsize Stassen (\citeyear{stassen1985comparison}), Kennedy (\citeyear{kennedy2005variation}), Bochnak (\citeyear{bochnak2013crosslinguistic})\normalsize
\end{itemize}
\end{itemize}

\ex. John exceeds Mary in height.

\begin{itemize}
\item research on Slavic prefixes and prepositions\\
\scriptsize Janda (\citeyear{janda1985meaning}), Filip (\citeyear{filip2000quantization}, \citeyear{filip2008events}), Ramchand (\citeyear{ramchand2004time}), Součková (\citeyear{souckova2004there}), Svenonius (\citeyear{svenonius2004slavic}), Romanova (\citeyear{romanova2006constructing}), Tatevosov (\citeyear{tatevosov2004slavic})\normalsize
\vspace{0.666em}
\begin{itemize}
\item scalarity in the verbal domain\\
\scriptsize Hay et al. (\citeyear{hay_kennedy_levin1999scalar}), Kearns (\citeyear{kearns2007telic}), Kennedy \& Levin (\citeyear{kennedy_levin2008measure}), Rappaport Hovav (\citeyear{rappaport-hovav2008lexicalized}), Beavers \& Koontz-Garboden (\citeyear{beavers_koontz-garboden2012manner}), Kagan (\citeyear{kagan2013scalarity})\normalsize
%HANA More representative references needed here: it is odd to have just the references to Rappaport Hovav (2011) and Kagan (2013); we should at least have Hay et al 1999, Kennedy and Levin 2008, Kearns 2007, Rappaport Hovav 2008, and also (again moi-meme) Filip 2008, given that Kagan 2013 builds on Filip 2008, Beavers 2011 and possibly other more recent Beavers references
\end{itemize}
\end{itemize}

\end{frame}

\begin{frame}{Introduction}

Empirical goal

\begin{itemize}
\item to examine two classes of Czech \textsc{exceed} verbs
%HANA \textit{exceed} verbs --> \textsc{exceed} verbs : caps used to emphasize that we mean a conceptual V class
%HANA both only seem to be prefixed, right?  If so then, we should have a small morphological note about itand skip "prefixed" in the headers, also added the part of speech class 
\begin{itemize}
%HANA \item prefixed deadjectival --> deadjectival
\item deadjectival
\end{itemize}
\end{itemize}
%HANA below dunno how to do the proper alignment
\exg. {\textcolor{blue}{vys}-oký  $\Rightarrow$} pře-\textcolor{blue}{vyš}-ovat\\
high-\textsc{adj} over-heighten-\textsc{ipf}\\
`high' $\Rightarrow$ `to exceed/be taller/higher (than)'

\begin{itemize}
\item[]
\begin{itemize}
%HANA \item prefixed non-deadjectival
\item non-deadjectival
\end{itemize}
\end{itemize}
%HANA we should also have the (potentially) initial N "krok" as the first derivation step or at least closely related to the verb "kracet"; not sure what the Slavic etymologists/morphologists think about it, so put in brackets, but it should be included;  btw according to wiki interesting etymology: Etymology From Old Norse krókr, from Proto-Germanic *krōkaz, from Proto-Indo-European *gerg-.
\exg. ({\textcolor{blue}{krok} $\Rightarrow$}) {\textcolor{blue}{kráč}-et $\Rightarrow$} pře-\textcolor{blue}{krač}-ovat\\
step-\textsc{n} step-\textsc{ipf} over-step-\textsc{ipf}\\
`step' $\Rightarrow$ `to step' $\Rightarrow$ `to exceed/overstep/transgress'

\begin{itemize}
%HANA hedging below: Are there ANY unprefixed EXCEED verbs? Are there any other prefixes ever used to derive EXCEED verbs?  What about před- in předběhnout or nad- in nadběhnout - are these EXCEED verbs and are these also prefixes used to form EXCEED verbs? These are also questions we may expect from the FASL attendees; in Russian, there is EXCEED "pere-" ...
\item both types %of \textsc{exceed} verbs in Czech 
are prefixed, most commonly with %the prefix 
\textit{pře-}
\end{itemize}
\end{frame}

\begin{frame}{Introduction}

Theoretical goal

\begin{itemize}
\item to integrate independent strands of research
\begin{itemize}
\item Slavic prefixes and prepositions and scales\\
\scriptsize Filip (\citeyear{filip2000quantization}, \citeyear{filip2008events}), Kagan (\citeyear{kagan2013scalarity})\small
\item class A/B numeral modifiers\\
\scriptsize Nouwen (\citeyear{nouwen2008directionality}, \citeyear{nouwen2010two}, \citeyear{nouwen2015modified})\small
\item comparative semantics\\
\scriptsize von Stechow (\citeyear{von_stechow1984comparing}), Heim (\citeyear{heim2000degree}), Schwarzschild (\citeyear{schwarzschild2008semantics})\normalsize
\end{itemize}
\end{itemize}

Claim

\begin{itemize}
\item prefix \textit{pře-} in Czech \textit{exceed} verbs
\begin{itemize}
\item class A comparative modifier
\item interaction with different stems $\Rightarrow$ predictable contrasts
\end{itemize}

\end{itemize}


\end{frame}

\begin{frame}{Introduction}

Outline

\begin{enumerate}
\item Introduction
\item Class A/B numeral modifiers
\item \textit{Exceed} comparison
\item Czech \textsc{exceed} comparatives
% \begin{itemize}
%HANA below changed italics to caps to emphasize that we mean a type of construction
% \item Czech \textit{exceed} constructions
% \end{itemize}
\item Slavic verb prefixes and scalarity
\item Proposal
\item Consequences% and further evidence
\item Modal constructions
\item Conclusion
\end{enumerate}

\end{frame}

\begin{frame}{Class A/B numeral modifiers}

Numeral modifiers as degree modifiers\\
\scriptsize Nouwen (\citeyear{nouwen2010two}); see also Brasoveanu (\citeyear{brasoveanu2012modified}) for a different framework\normalsize

\begin{itemize}
\item class A $\Rightarrow$ comparative modifiers
\item class B $\Rightarrow$ maxima/minima indicators% superlative modifiers 
\end{itemize}

% Please add the following required packages to your document preamble:
% \usepackage{booktabs}
\begin{table}[]
\centering
%\caption{My caption}
\label{table:class-AB}
\begin{tabular}{@{}ll@{}}
\toprule
\multicolumn{1}{c}{A}    & \multicolumn{1}{c}{B}\\ \midrule
more than \textit{n}     & at least \textit{n}  \\
less than \textit{n}     & at most \textit{n}   \\
fewer than \textit{n}    & up to \textit{n}     \\
over \textit{n}          & from \textit{n}      \\
under \textit{n}         & from \textit{n} to \textit{m} \\
between \textit{n} and \textit{m} & minimally \textit{n} \\
                & maximally \textit{n} \\ \bottomrule
\end{tabular}
\end{table}

% \ex. under/over/more than/fewer than 100

% \ex. at least/at most/up to/maximally 100

\end{frame}

\begin{frame}{Class A/B numeral modifiers}

Diagnostics\\
\scriptsize Nouwen (\citeyear{nouwen2010two})\normalsize

\begin{itemize}
\item class A: can assert extremely weak propositions
\item class B: cannot express relations to definite amounts \\
\footnotesize (except when embedded under an existential modal)\normalsize 
\end{itemize}

\ex. \a. A hexagon has \textcolor{blue}{fewer than 11} sides.
\b. A hexagon has \textcolor{blue}{more than 3} sides.
%HANA  I'd change it to "at most/least 6 sides". I.e., I know that a hexagon has exactly 6 sides, therefore to say "A hexagon has at most/least 6 sides" is odd. If I don't know that a hexagon has exactly 6 sides then both (5a) and (5b) are acceptable. This needs to be mentioned immediately, rather than mentioned on the next slide.
% ALSO Nouwen (2010): B "acceptable when what is ‘under discussion’ is not a definite amount, but rather a range of amounts"

Context: I know that a hexagon has exactly 6 sides.

\ex. \a. \#A hexagon has \textcolor{red}{at most 10} sides.
\b. \#A hexagon has \textcolor{red}{at least 3} sides.

\end{frame}

\begin{frame}{Class A/B numeral modifiers}

Ignorance effects\\
\scriptsize Nouwen (\citeyear{nouwen2015modified})\normalsize

%HANA "epistemic competence" - I assume that you take it in the technical sense in which philosophers use it. If so, in the paper (and also in the handout, if you wish) we should refer to Sosa (2007), for instance: "Sosa (2007) claims that a necessary condition on knowledge is manifesting an epistemic competence. To manifest an epistemic competence, a belief must satisfy two conditions: (1) it must derive from the exercise of a reliable belief-forming disposition in appropriate conditions for its exercise and (2) that exercise of the disposition in those conditions would not issue a false belief in a close possible world."
\begin{itemize}
\item class A: compatible with epistemic competence
%HANA "incompatible $\Rightarrow$ indicate ignorance" --> license ignorance inference
\item class B: license ignorance inference 
\end{itemize}

\ex. \a. I have \textcolor{blue}{more than 2} children.
\b. I have \textcolor{red}{at least 3} children.

\ex. \a. There were exactly 62 mistakes in the manuscript, so that's \textcolor{blue}{more than 50}.
\b. There were exactly 62 mistakes in the manuscript, \#so that's \textcolor{red}{at least 50}.

%HANA in the context in which I'm looking for at least 50 mistakes, say, because I am a nasty grader and want to fail a student, (7b) is perfectly fine.
\end{frame}

\begin{frame}{\textsc{Exceed} comparison}

\textsc{Exceed} comparatives\\
\scriptsize Stassen (\citeyear{stassen1985comparison})\normalsize

\begin{itemize}
\item standard of comparison $\Rightarrow$ direct object of a special transitive verb $\approx$ `exceed' or `surpass'
\item examples: Mandarin, Vietnamese, Swahili, Yoruba, Hausa
\end{itemize}

\exg. Ta \textcolor{blue}{bi} ni gau.\label{ex:exceed-mandarin}\\
he exceed you tall\\
`The is taller than you.'\hfill \footnotesize Mandarin\normalsize

\exg. O tobi \textcolor{blue}{ju} u.\label{ex:exceed-yoruba}\\
he big exceed him\\
`He is bigger than him.'\hfill \footnotesize Yoruba \normalsize

\end{frame}

\begin{frame}{\textsc{Exceed} comparison}

\textsc{Exceed} comparatives\\
\scriptsize Stassen (\citeyear{stassen1985comparison})\normalsize

\begin{itemize}
\item sub-types
\begin{itemize}
\item main predicate of a comparative sentence
\item subordinate nominalized form
\item serial-verb constructions
\end{itemize}
\item \textsc{exceed} comparatives can co-exist with other strategies
%HANA \item \textsc{exceed} constructions can co-occur with other strategies
%HANA \item \textsc{exceed} comparatives can be expressed by other means
\item examples: English, Czech
\end{itemize}
%HANA geez - I am confused here.  (10) and (11) are examples of "main predicate of a comparative sentence", I think. If so, the heading "EXCEED comparatives can be expressed by other means" is misleading, or this heading should be the 4th item under "sub-types"
\ex. John's height \textcolor{blue}{exceeds} Mary's height.\label{ex:exceed-english}

\exg. Katedrála \textcolor{blue}{převyšuje} radnici.\label{ex:exceed-czech}\\
cathedral exceeds-in-height town-hall\\
`The cathedral exceeds the town-hall in height.'\hfill \footnotesize Czech \normalsize

\end{frame}

\begin{frame}{Czech \textsc{exceed} comparatives}

Czech \textsc{exceed} verbs

\begin{itemize}
\item two classes
\begin{itemize}
\item deadjectival: \textit{pře-vyš-ovat} (lit. `over-heighten')
%HANA (lit. `over-high') --> (lit. `over-heighten')
\item non-deadjectival: \textit{pře-krač-ovat} (lit. `over-step')
\end{itemize}
\item comparative nature
\begin{itemize}
\item compare objects with respect to some dimension
\item compatible with differentials
\end{itemize}
\end{itemize}

\exg. Katedrála \textcolor{blue}{pře-vyš-uje} radnici \textcolor{blue}{o} \textcolor{blue}{20} \textcolor{blue}{metrů}.\label{ex:ExV-diff}\\
cathedral over-high-\textsc{ipf.3.sg} {town-hall} by 20 meters\\
`The cathedral exceeds the town hall in height by 20 meters.'
%`The cathedral is 20 meters higher than the town hall.'

\end{frame}

\begin{frame}{Czech \textsc{exceed} comparatives}
%HANA "constructions" --> comparatives (more neutral) because we also refer to verbs

Deadjectival \textsc{exceed} verbs

\begin{itemize}
\item derived from comparative forms of adjectives
\begin{itemize}
\item consonantal alternations in comparatives
\item suppletive morphology
\end{itemize}
\end{itemize}

\ex. \ag. {vy\textcolor{red}{s}-oký $\sim$} vy\textcolor{blue}{š}-ší\\
high higher\\
\bg. *{pře-vy\textcolor{red}{s}-ovat $\sim$} pře-vy\textcolor{blue}{š}-ovat\\
over-high-\textsc{ipf} over-higher-\textsc{ipf}\\

\ex. \ag. {\textcolor{red}{dobr}-ý $\sim$} \textcolor{blue}{lepš}-í\\
good better\\
\bg. *{po-\textcolor{red}{dobř}-it $\sim$} po-\textcolor{blue}{lepš}-it\\
\textsc{po}-good-\textsc{pfv} \textsc{po}-better-\textsc{pfv}\\
%`improve'

\end{frame}

\begin{frame}{Czech \textsc{exceed} comparatives}

Deadjectival \textsc{exceed} verbs

\begin{itemize}
\item prefix $\Rightarrow$ obligatory part of the derivation
\begin{itemize}
\item all deadjectival \textsc{exceed} verbs seem to be prefixed
\item ungrammaticality of primary (im)perfectives
\end{itemize}
\end{itemize}

%HANA in the example below: changed the gloss "high-" to "higher-", because on the previous slide we emphasize that the base from which the verb is formed is the comparative 
\exg. *{výš-it $\sim$} *{vyš-ovat $\sim$} {\textcolor{blue}{pře}-výš-it $\sim$} \textcolor{blue}{pře}-vyš-ovat\\
higher-\textsc{pfv} higher-\textsc{ipf} over-higher-\textsc{pfv} ovr-high-\textsc{ipf}\\

Non-deadjectival \textsc{exceed} verbs

\begin{itemize}
\item derived from verbs of motion %překračovat, přesahovat, 
\end{itemize}

\exg. {\textcolor{blue}{kráč}-et $\Rightarrow$} pře-\textcolor{blue}{krač}-ovat\\
step-\textsc{ipf} over-step-\textsc{ipf}\\
`step' $\Rightarrow$ `exceed'

\end{frame}

\begin{frame}{Czech \textsc{exceed} comparatives}

Class A/B distinction

\begin{itemize}
\item \textsc{exceed} verbs pattern with class A modifiers
\item can express relations to definite cardinalities
\item can assert extremely weak propositions
\end{itemize}

\exg. Počet stran šestiúhelníku \textcolor{blue}{pře-kračuje} 3.\\
number of-sides of-hexagon over-steps 3\\
`The number of sides of a hexagon exceeds 3.'

\exg. \#Počet stran šestiúhelníku je \textcolor{red}{od} 5 \textcolor{red}{do} 7.\\
number of-sides of-hexagon is from 5 to 7\\

\end{frame}

\begin{frame}{Czech \textsc{exceed} comparatives}

Ignorance effects

\begin{itemize}
\item class B modifiers $\Rightarrow$ ignorance inference
\item referentially determined but epistemically uncertain (various explanations: \cite{nouwen2010two} vs. \cite{buring2008least}) 
%\\\footnotesize cf. Nouwen (2015)\normalsize
%HANA \item no ignorance effects with prefixed \textit{exceed} verbs 
%HANA \item prefixed \textsc{exceed} verbs pattern with class A modifiers: do not license ignorance inference
\end{itemize}
%HANA "#" might be weakened to "(#)", i.e., in the context of a contest, e.g., "The Apprentice" say, 
% my task is to find an apartment in Prague which is between 100.000 and 200.000 Euro. I succeed, and so I can felicitously utter (19). Out of the blue (19) might just a bit odd.
\exg. Cena toho bytu byla 120.000 {\euro,} \textcolor{red}{(\#)}takže byla \textcolor{red}{od }~{ }~100.000 \textcolor{red}{do} {200.000 {\euro}}.\label{ex:classB}\\
price of-this flat was 120.000 {\euro} so was from~100.000 to {200.000 {\euro}}\\

\end{frame}

\begin{frame}{Czech \textsc{exceed} comparatives}

\begin{itemize}
\item \textsc{exceed} verbs/class A modifiers $\Rightarrow$ no ignorance effects
\end{itemize}

\exg. Cena toho bytu byla 120.000 {\euro,} takže \textcolor{blue}{pře-kročila} {100.000 {\euro}}.\label{ex:classA}\\
price of-this flat was 120.000 {\euro} so over-stepped.\textsc{pfv} {100.000 {\euro}}\\
`The price of this flat was 120.000 {\euro}, so it exceeded 100.000 {\euro}.'

% \ex. \a. Cena toho domu byla 1200 EUR, takže pře-kročila 1000 EUR.\\
%  price that.GEN house.GEN was 1200 EUR so over-reached 1000 EUR
% \b. Cena toho domu byla 1200 EUR, \# takže byla od 1000 EUR do 2000 EUR.\\
% price that.GEN house.GEN was 1200 EUR so was from 1000 EUR to 2000 EUR

\end{frame}

\begin{frame}{Czech \textsc{exceed} comparatives}

Class B ignorance implicature disappears

\begin{itemize}
\item in the scope of certain modals
\item in variation readings (e.g., generic context)
\end{itemize}

\exg. Cena toho bytu může být \textcolor{blue}{od} 100.000 \textcolor{blue}{do} {200.000 {\euro}}.\label{ex:classB-modal}\label{ex:classB-modal-variation}\\
price of-this flat can be from 100.000 to {200.000 {\euro}}\\
`The price of this flat can be from 100.000 to 200.000~{\euro}.'

\exg. Ceny bytů tu jsou \textcolor{blue}{od} 100.000 \textcolor{blue}{do} {200.000 {\euro}}.\label{ex:classB-variation}\\
prices flats here are from 100.000 to {200.000 {\euro}}\\
`The prices of flats here can be from 100.000 to 200.000 {\euro}.'

% \ex. \a. Cena toho domu může být od 1000 do 2000 EUR.\\
% price that.GEN house.GEN can be from 1000 to 2000 EUR
% \b. Ceny domů tu jsou od 1000 EUR do 2000 EUR.\\
% prices that.GEN houses.GEN here are from 1000 EUR to 2000 EUR

\end{frame}

\begin{frame}{Slavic verb prefixes and scalarity}

Measures\\\scriptsize Filip (\citeyear{filip1992aspect}, \citeyear{filip2000quantization})\normalsize

\begin{itemize}
\item majority of Slavic verb prefixes 
\begin{itemize}
	\item multiple uses
	\item common general meaning 
\end{itemize}
%that can be semantically analyzed in terms of (extensive) measure functions over (i) the temporal trace or path of eventualities, or some measurable dimension of their participants (possibly also including affective aspects concerning their effort, (dis)satisfaction and the like); or (ii) pluralities of eventualities.
\item extensive measure functions over
\begin{itemize}
\item temporal trace or path of eventualities
\item some measurable dimension of their participants% (possibly also including affective aspects concerning their effort, (dis)satisfaction and the like)
\item pluralities of eventualities
\end{itemize}
\end{itemize}
%Inspired by Nouwen, we recast the core scalar contribution of Slavic verbal prefixes in terms of a type A (locative P) vs. type B (directional P) modifiers.

\end{frame}

\begin{frame}{Slavic verb prefixes and scalarity}

Extension to scales\\\scriptsize Filip (\citeyear{filip2005measure}, \citeyear{filip2008events})\normalsize

\begin{itemize}
\item verb prefixes
\begin{itemize}
\item introduce measure functions
\item measure functions are mapped to degrees on scales 
\item scales represent the measure of the relevant entities wrt the corresponding dimension
\end{itemize}
\item study of Russian prefixes\\\scriptsize Kagan (\citeyear{kagan2011scale}, \citeyear{kagan2012degree}, \citeyear{kagan2013scalarity}), Kagan \& Alexeyenko (\citeyear{kagan_alexeyenko2011adjectival})\small
\begin{itemize}
\item extensive scale-based analysis
\item unified semantics
\end{itemize}

% \item Kagan (2013, and elsewhere): building on Filip (2008), i.a., provides an extensive scale-based analysis of Russian verb prefixes.
\end{itemize}

\end{frame}

	\begin{frame}{Proposal}
		
		Assumptions regarding comparison
		
		\begin{itemize}
			\item ontology: degrees (type $d$) ordered into scales
			\item scale: $\langle D, >, DIM\rangle$ 
			\begin{itemize}
				\item $D$: a set of degrees
				\item $>$: an ordering relation on $D$
				\item $DIM$: a dimension of measurement, e.g., height
			\end{itemize}
			\item interval-based approach to degrees\\\scriptsize Kennedy (\citeyear{kennedy2001polar}), Schwarzschild \& Wilkinson (\citeyear{schwarzschild_wilkinson2002quantifiers})\normalsize
			\item measure functions associate entities with scales\\\scriptsize Solt (\citeyear{solt2015q-adjectives})\normalsize
%HANA Is it the JoS paper  - Q-adjectives and the semantics of quantity - S Solt Journal of semantics 32 (2), 221-273 if so it should be 2015, 
			\item semantics of gradable adjectives
            \item comparative semantics: A $>$ B\\\scriptsize von Stechow (\citeyear{von_stechow1984comparing}), Heim (\citeyear{heim2000degree}), Schwarzschild (\citeyear{schwarzschild2008semantics})\normalsize
		\end{itemize}

\end{frame}

\begin{frame}{Proposal}

Prefix \textit{pře-} in \textsc{exceed} verbs 
%HANA again I'd use italics for example tokens, and use small caps for conceptual classes
\begin{itemize}
\item comparative degree quantifier
\begin{itemize}
\item class A modifier
\item built-in \textsc{min} operator\\\scriptsize cf. Hackl (\citeyear{hackl2001comparative})\normalsize
\end{itemize}
\item adjectival positive stem 
\begin{itemize}
\item appropriate scale $\Rightarrow$ dimension $DIM$: height etc.
\end{itemize}
%HANA "comparative form", i.e., of the adjectival base, right?  So I added "adjectival" but not sure whether it is right
\item adjectival comparative form
\begin{itemize}
\item \textsc{max} $\Rightarrow$ definite description of a maximal degree
\item > relation
\end{itemize}

\end{itemize}

\ex. $\llbracket\textit{pře-~d}\rrbracket = \lambda M.\textsc{min}_{d'}(M(d'))>d$\label{ex:pre-semantics} 

\end{frame}

\begin{frame}{Proposal}

Composition

\begin{itemize}
\item \textsc{min} operates `on top of' the comparative semantics
\item \textsc{max} $\Rightarrow$ the maximal degree within an interval on a scale
\item \textit{pře-}: \textsc{min} $\Rightarrow$ the very same value and the result is \\
comparison with the degree corresponding to a correlate
\end{itemize}

\exg. Katedrála pře-vyš-uje radnici.\label{ex:ExVs}\\
%HANA  changed "-high-" to "-higher-, because it is based on the comparative ADJ base
cathedral over-higher-\textsc{ipv.3.sg} town-hall\\
`The cathedral exceeds the town-hall in height.'
%HANA I don't understand why we need MAXd in MAXd.lambda_d ... town-hall ... 
% HANA I had to cite it out, because for some reason the file did not compile with this formula: 

\ex. $\textsc{min}_{d'}(\textsc{max}_{d'}(\lambda d'[\mu_{\textsc{height}}(\textsc{cathedral})\geq d']))>\textsc{max}_{d}(\lambda d[\mu_{\textsc{height}}(\textsc{town-hall})\geq d])$\label{ex:exceed-semantics} 


\end{frame}


\begin{frame}{Proposal}

Differentials

\begin{itemize}
\item differential comparative: $\geq$ + additional degree argument\\
\scriptsize cf. von Stechow (\citeyear{von_stechow1984comparing}), Beck (\citeyear{beck2011comparison})\normalsize
\item additional degree measures the gap between the maxima of the standard and correlate\\
\scriptsize cf. Schwarzschild (\citeyear{schwarzschild2008semantics})\normalsize
\end{itemize}

\exg. Katedrála pře-vyš-uje radnici o 20 m.\\
%HANA  changed "-high-" to "-higher-, because it is based on the comparative ADJ base
cathedral over-higher-\textsc{ipv.3.sg} town-hall by 20 m\\
`The cathedral exceeds the town-hall in height by 20m.'
% HANA I had to cite it out, because for some reason the file did not compile with this formula:  

\ex. $\textsc{min}_{d'}(\textsc{max}_{d'}(\lambda d'[\mu_{\textsc{height}}(\textsc{cathedral})\geq d']))\geq \textsc{max}_{d}(\lambda d[\mu_{\textsc{height}}(\textsc{town-hall})\geq d]) + \textsc{20meters}$\label{ex:exceed-differential-semantics} 


\end{frame}

\begin{frame}{Proposal}

Class B modifiers\\
\scriptsize Nouwen (\citeyear{nouwen2010two}) or \cite{brasoveanu2012modified}\normalsize

\begin{itemize}
\item minima/maxima indicators: \textit{minimally}, \textit{maximally}, ...
\item often directional Ps: \textit{od}\dots \textit{do} (`from\dots to')
\item epistemic uncertainty 
%\item pragmatic mechanism $\Rightarrow$ licenses ignorance inference
%HANA "ignorance effects" replaced by "license ignorance inference"
\end{itemize}

\ex. \a. $\llbracket\textit{minimally~d}\rrbracket=\lambda M.\textsc{min}_{d'}(M(d'))=d$
\b. $\llbracket\textit{maximally~d}\rrbracket=\lambda M.\textsc{max}_{d'}(M(d'))=d$

\ex. \a. $\llbracket\textit{od~d}\rrbracket=\lambda M.\textsc{min}_{d'}(M(d'))=d$
\b. $\llbracket\textit{do~d}\rrbracket=\lambda M.\textsc{max}_{d'}(M(d'))=d$

\end{frame}

\begin{frame}{Consequences}

Prefix \textit{pře-} as a class A modifier 

\begin{itemize}
\item requires a value on a scale to employ the > relation
\item degree-denoting argument NP $\Rightarrow$ value of the standard  
\item subject NP $\Rightarrow$ appropriate dimension  
\item \textsc{result}: both types of Czech \textsc{exceed} verbs are felicitous
\end{itemize}

\exg. Teplota \textcolor{blue}{pře-vyš-uje} \textcolor{blue}{20\degree C}.\label{ex:ExVs-degrees}\\
temperature over-higher-\textsc{ipv.3.sg} 20\degree C\\
`The temperature exceeds 20\degree C.'

\exg. Teplota \textcolor{blue}{pře-krač-uje} \textcolor{blue}{20\degree C}.\\
temperature over-step-\textsc{ipv.3.sg} 20\degree C\\
`The temperature exceeds 20\degree C.'

\end{frame}

\begin{frame}{Consequences}

\begin{itemize}
\item standard is not degree-denoting $\Rightarrow$ asymmetry
\item deadjectival \textsc{exceed} verbs associate an entity with an interval on a scale
\item \textsc{result}: \textit{pře-} can apply the comparative semantics
\item non-deadjectival \textsc{exceed} verbs fail to relate an entity with a scale
\item \textsc{result}: > cannot be applied $\Rightarrow$ derivation crashes
\end{itemize}

\exg. Katedrála \textcolor{blue}{pře-vyš-uje} \textcolor{blue}{radnici}.\label{ex:ExVs-entities}\\
cathedral over-higher-\textsc{ipv.3.sg} town-hall\\
`The cathedral exceeds the town-hall in height.'

\exg. *Katedrála \textcolor{red}{pře-krač-uje} \textcolor{red}{radnici}.\\
cathedral over-step-\textsc{ipv.3.sg} town-hall\\

\end{frame}

\begin{frame}{Consequences}

\begin{itemize}
\item \textsc{important distinction}: individuals vs. degrees
\item degree nominals $\Rightarrow$ degree semantics \\\scriptsize cf. Morzycki (\citeyear{morzycki2009degree})\normalsize
\item standard with a degree semantics $\Rightarrow$ non-deadjectival \textsc{exceed} verbs are felicitous
\end{itemize}
%HANA notice also that we have here a metaphoric use of "překračuje"
%interestingly "převyšuje" is not often used metaphorically: e.g. "to převyšuje moje schopnosti", it's not easy to find such examples
\ex.\label{ex:ExVs-degree-nouns} \ag. To pře-krač-uje moje očekávání.\\
this over-step-\textsc{ipv.3.sg} my expectations\\
`This exceeds my expectations.'
\bg. To pře-krač-uje všechny meze.\\
this over-step-\textsc{ipv.3.sg} all limits\\
`This exceeds all limits.'

\end{frame}

\begin{frame}{Consequences}

\begin{itemize}
\item \textsc{further prediction}: no \textsc{exceed} verbs with negative class A prefixes
\item \textsc{reason}: \textit{pod-} reverses the scale $\Rightarrow$ < relation conflicts with > semantics of the adjectival stem
\item \textsc{result}: inevitable contradiction $\Rightarrow$ ungrammaticality\\\scriptsize Gajewski (\citeyear{gajewski2002analycity})\normalsize
\end{itemize}

\exg. *{\textcolor{red}{pod}-výš-it $\sim$} *\textcolor{red}{pod}-vyš-ovat\\
under-higher-\textsc{pfv} under-higher-\textsc{ipv}\\

\end{frame}

\begin{frame}{Consequences: class B modifiers}

\begin{itemize}
\item basic case: \Next[a], 'floated' modifier in \Next[b], Slavic prefixal/prepositional numeral modification in \NNext
\end{itemize}

\ex. \a. At most three boys saw exactly five movies.
\b. Three boys saw five movies, exactly/at (the) most.

\exg. denní teploty do-sáhnou 19 až 23\degree C\\
day temperatures to-reach-\textsc{pf.3.pl} 19 to 23\degree C\\
`The day temperatures will reach to 19 to 23\degree C.'


\end{frame}

\begin{frame}{Consequences: class B modifiers}

\begin{itemize}
\item modified numerals or degrees
\item cumulative readings observed with modified numerals detected: \Next
\item the right scoping mechanism: \cite{brasoveanu2012modified} -- numerals as post-suppositions
\end{itemize}

\exg. Ceny těch tří bytů do-sáhly 100, 200 a 300 tisíc Eur.\\
prices the three flats to-reach 100 200 and 300 thousand Eur\\
`The prices of the three flats reached to 100, 200 and 300 thousands of Euro.'
\a. a \ldots 100 000
\b. b \ldots 200 000
\c. c \ldots 300 000

\end{frame}

\begin{frame}{Modal constructions}

Scope of existential modals

\begin{itemize}
\item class A modifiers $\Rightarrow$ ambiguity
\begin{itemize}
\item weak reading: $\diamond$ > 100.000 $\Rightarrow$ true both in $< d$ and $\geq d$
\item strong reading: 100.000 > $\diamond$ $\Rightarrow$ true only in $< d$ 
\end{itemize}
\end{itemize}

\exg. Ten byt můžeš prodat \textcolor{blue}{pod} 100.000 {\euro}.\label{ex:classA-ambiguity}\\
this flat you-can sell.\textsc{pfv} under 100.000 {\euro}\\
`You can sell this flat for less than 100.000 {\euro}.'
\a. $\diamond[\textsc{max}_d(\exists !x[\textsc{price}(x,d) \wedge \textsc{flat}(x) \wedge \textsc{sell}(you,x)]) < 100.000]$\label{ex:classA-weak}
\b. $\textsc{max}_d(\diamond \exists ! x[\textsc{price}(x,d) \wedge \textsc{flat}(x) \wedge \textsc{sell}(you,x)]) < 100.000$\label{ex:classA-strong}

% \exg. Ten byt můžeš prodat nad 100.000 {\euro}.\label{ex:classA-ambiguity}\\
% this flat you-can sell.pf over 100.000 {\euro}\\
% \a. $\diamond[\textsc{max}_d(\exists !x[\textsc{price}(x,d) \wedge \textsc{flat}(x) \wedge \textsc{sell}(you,x)]) > 100.000]$\label{ex:classA-weak}
% \b. $\textsc{max}_d(\diamond \exists ! x[\textsc{price}(x,d) \wedge \textsc{flat}(x) \wedge \textsc{sell}(you,x)]) > 100.000$\label{ex:classA-strong}

\end{frame}

\begin{frame}{Modal constructions}

Scope of existential modals

\begin{itemize}
\item class B modifiers $\Rightarrow$ no ambiguity
\begin{itemize}
\item only strong reading: 100.000 > $\diamond$ $\Rightarrow$ true only in $< d$ 
\end{itemize}
\end{itemize}

\exg. Ten byt můžeš prodat \textcolor{red}{až} \textcolor{red}{do} 100.000 {\euro}.\\
this flat you-can sell.\textsc{pfv} up to 100.000 {\euro}\\
`You can sell this flat for up to 100.000 {\euro}.'
\a. $\textsc{max}_d(\diamond \exists ! x[\textsc{price}(x,d) \wedge \textsc{flat}(x) \wedge \textsc{sell}(you,x)])=100.000$\label{ex:classB-strong}

\end{frame}

\begin{frame}{Modal constructions}

Scope of existential modals

\begin{itemize}
\item maximal permissions with class B
\item the numeral in the object obligatorily "outscopes" the modal verb
\item Googled example: strong support for Brasoveanu's post-supposition treatment of modified numerals in \NNext
\end{itemize}

\exg. jezevčík velký \ldots  váha smí do-sáhnout 10 kg\\
dachshund big \ldots weight can to-reach 10 kg\\
`The weight of dachshund big is allowed to reach to 10 kg.'

\ex. $\textsc{max}([w]\wedge R_{*w}(w) [x] \wedge x=\textsc{weight}(Dachshund,d)) \wedge ^{d \leq 10}$

\end{frame}

\begin{frame}{Universal quantifier and class B}

\begin{itemize}
\item cumulative vs. obligatory distributive reading?
\end{itemize}

\ex. \a. Tři elektárny překročily všechny limity, elektrárna A v těžbě uhlí, elektrárna B v \ldots, elektrárna C v množství radioaktivity.
\b. Všechny elektrárny překročily tři limity, \# elektrárna A v těžbě uhlí, elektrárna B v \ldots, elektrárna C v množství radioaktivity.

\end{frame}

\begin{frame}{Modal constructions}

Scope of existential modals

\begin{itemize}
\item \textsc{exceed} verbs modifiers $\Rightarrow$ no ambiguity
\begin{itemize}
\item only weak reading: $\diamond$ > 100.000 $\Rightarrow$ true in $< d$ and $\geq d$ 
\end{itemize}
\end{itemize}

\exg. Cena toho bytu může \textcolor{gray}{překročit} 100.000 {\euro}.\label{ex:classA-prefix}\\
price of-this flat can over-step.\textsc{pfv} 100.000 {\euro}\\
`The price of this flat can exceed 100.000 {\euro}.'
\a. $\diamond[\textsc{min}_d(\exists !x[\textsc{price}(x,d) \wedge \textsc{flat}(x) \wedge \textsc{sell}(you,x)]) > 100.000]$

\end{frame}

\begin{frame}{Modal constructions}

Scope patterns

\begin{itemize}
\item \textsc{exceed} verbs
\begin{itemize}
\item opposite pattern wrt class B, but discrepancy wrt class A
\end{itemize}
\end{itemize}

\begin{table}[]
\centering
%\caption{My caption}
\label{table-scope-a-b-exceed}
\begin{tabular}{@{}llll@{}}
\toprule
                          & A      & B      & \textit{exceed}  \\ \midrule
$\diamond > d$ & $\checkmark$ & *      & $\checkmark$ \\
$d > \diamond$ & $\checkmark$ & $\checkmark$ & *      \\ \bottomrule
\end{tabular}
\end{table}

\begin{itemize}
\item possible explanation
\begin{itemize}
\item structural restrictions rather than a different semantics
\item morpho-syntactic status of the prefix
\item \textit{pře-} as part of the verbal morphology $\Rightarrow$ structurally predetermined to stay in the scope of the modals (but why not in case of class B?)
\end{itemize}

\end{itemize}


% consequence of the morpho-syntactic status of the prefix, i.e., \textit{pře-} being part of the verbal morphology is structurally predetermined to stay in the scope of the operators that outscope the ExV, such as the modal in \ref{ex:classA-prefix}.~But this means that the unexpected reading of \ref{ex:classA-prefix} results from structural restrictions rather than from a different semantics.~The above data have so far not been noticed in the typology of the grammar of comparison, and provide a direct empirical support for Nouwen's (2010, 2015) hypothesis that the class A/class B distinction may indeed be cross-linguistically, if not possibly universally, valid.

\end{frame}
		
	\begin{frame}{Conclusion}
		
		Observations

		\begin{itemize}
			\item novel data wrt typology of the grammar of comparison
            \item \textsc{exceed} comparison in Czech
			\begin{itemize}
				\item deadjectival \textit{exceed} verbs
				\item non-deadjectival \textit{exceed} verbs
			\end{itemize}
			\item Czech \textsc{exceed} verbs pattern with class A modifiers
			\begin{itemize}
				\item can relate to definite cardinalities
				\item no ignorance effects
			\end{itemize}
			\item class A/B generalization  
\begin{itemize}
\item empirical support $\Rightarrow$ cross-linguistically valid
\end{itemize}
		\end{itemize}
		
	\end{frame}
	
	\begin{frame}{Conclusion}
		
		Proposal
		
		\begin{itemize}
			\item Czech prefix \textit{pře-} in \textsc{exceed} verbs    
			\begin{itemize}
				\item class A semantics
\item degree quantifier
\item comparative meaning 
			\end{itemize}
            \item consequences
\begin{itemize}
\item interaction of \textit{pře-} with (non)deadjectival stems 
\item predictable contrasts between the two classes
\item no \textsc{exceed} verbs with negative class A prefixes
\end{itemize}
		\end{itemize}
		
	\end{frame}
	
	\begin{frame}{Conclusion}
		
		Further research:
		
		\begin{itemize}
			\item \textsc{exceed} verbs in modal constructions
			\item unified analysis of \textit{pře-}\\
            \scriptsize cf. Kagan (\citeyear{kagan2013scalarity})\normalsize
			\begin{itemize}
				\item \textsc{exceed} verbs
				\item other types of verbs
			\end{itemize}            
            \item locative/directional Ps and class A/B mismatches
			\begin{itemize}
				\item \textit{pře-}: class A behavior
				\item \textit{přes}: directional P
			\end{itemize}
			\item modal constructions: emerging patterns ($\rightarrow$ experiments)
			\item cross-linguistic investigation
			\begin{itemize}
				\item within Slavic
				\item beyond
			\end{itemize}
		\end{itemize}
		
	\end{frame}
	
	\begin{frame}[t,allowframebreaks]{References}
%		\frametitle{References}
		\bibliography{\jobname.bib}

% \textbf{References.} 
% $\bullet$ Filip 2000. "The Quantization Puzzle." In \textit {Events as grammatical objects, from the combined perspectives of lexical semantics, logical semantics and syntax}, edited by Carol Tenny and James Pustejovsky. Stanford: CSLI Press. Pp.39-91.  
% $\bullet$ Filip 2008. “Events and Maximalization.” In \textit {Theoretical and Crosslinguistic Approaches to the Semantics of Aspect}, edited by Susan Rothstein. Amsterdam: John Benjamins. Pp.217-256. 
% $\bullet$ Gajewski (2002) On analyticity in natural languages
% $\bullet$ Hackl (2001) Comparative quantifiers 
% $\bullet$ Heim (2000) Degree operators and scope
% $\bullet$ Kagan (2013) Scalarity in the domain of verbal prefixes 
% $\bullet$ Nouwen (2008) Directionality in numeral quantifiers 
% $\bullet$ Nouwen (2010) Two kinds of modified numerals 
% $\bullet$ Nouwen (2015) Modified numerals: The epistemic effect 
% $\bullet$ Nouwen (2016) Making sense of the spatial metaphor for number in natural language 
% $\bullet$ Schwarzschild (2008) The semantics of comparatives and other degree constructions 
% $\bullet$ Stassen (1985) Comparison and universal grammar 

\end{frame}

	\begin{frame}
    
		\begin{center}
			{\Huge Thanks!}
		\end{center}
		
	\end{frame}
		
\end{document}